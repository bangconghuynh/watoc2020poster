% !TeX program = lualatex
% !TeX TXS-program:bibliography = txs:///biber


% ========
% Preamble
% ========

% -----------------------------
% Font, theme, and Beamerposter
% -----------------------------
\usepackage[orientation=portrait, size=a0, scale=1.1]{beamerposter}

\usetheme{BCHNottingham}

\usecolortheme{BCHNottingham}

\newcommand\bmmax{2}

\usepackage{bm}


% ---------
% Chemistry
% ---------
\usepackage[version=4]{mhchem} % Further chemistry typesetting support


% ----------------------
% Scientific typesetting
% ----------------------
\usepackage{siunitx}
\AtBeginDocument{\sisetup{math-rm=\symup, text-rm=\rmfamily}}


% ----------------
% Micro-typography
% ----------------
\usepackage[
  activate={true,nocompatibility},%
  final,%
  tracking=true,%
  factor=1100,%
  stretch=10,%
  shrink=10%
]{microtype}
\SetTracking{encoding={*}, shape=sc}{40} % Reduce spacing between sc characters
\microtypecontext{spacing=nonfrench}


% -------------
% Maths support
% -------------
\usepackage{amsmath, amssymb, mathtools, mathrsfs, braket, amsthm, interval} % packages for further maths support

\usepackage{mleftright} %loading package for ensuring correct spacing before brackets
\mleftright


% ------
% Tables
% ------
\usepackage{booktabs, multirow, tabularx}

\usepackage{lscape}


%% -------------
%% In-line lists
%% -------------
%\newenvironment{inlineitemize}{%
%  \let\par\relax%
%  \def\item{\usebeamertemplate{itemize item}\hspace{1mm}}
%  \leavevmode%
%}{}

% -----------------
% Cross-referencing
% -----------------
\usepackage[capitalise, noabbrev]{cleveref} %loading package for enhanced cross-referencing; cleverref must be loaded after hyperref


% ---------------------
% Diagrams and captions
% ---------------------
\usepackage{graphicx}

\usepackage{caption, subcaption}
\captionsetup{justification=centering}


% pgfplots
%% We need the external library to use the \tikzsetexternalprefix variable
%% in all cases. However, we only need version compatibility and the rest of
%% the libraries if we actually need to compile pikz diagrams.

\usepackage{pgfplots} % pgfplots loads tikz automatically

\usetikzlibrary{external}

\tikzexternalize

% \tikzexcmd is defined in the compilation call from the command line. See the Makefile.
\newif\iftikzex
\ifdefined\notikzex
  \tikzexfalse
\else
  \tikzextrue
\fi

\iftikzex
  \pgfplotsset{compat=1.17}

  \usetikzlibrary{calc, luamath, positioning, pgfplots.groupplots, arrows.meta}

  \pgfplotscreateplotcyclelist{coloronly}{%
    {red},%
    {Blue},%
    {black!60!green},%
    {black!20!orange},%
    {green!30!brown},%
    {Blue!40!red},%
    {black!60!Blue},%
    {black!40!yellow},%
    {red!50!pink},%
    {green!70!Blue},%
  }
\fi

% Tikzexternalize
\makeatletter
\newcommand*{\useexternalfile}[4]{
  \iftikzex
    \tikzsetnextfilename{tikzoutput/#4-output}
    \scalebox{#1}{\input{\tikzexternal@filenameprefix#4.tikz.tex}}
  \else
    \includegraphics[scale=#1, trim=#2 0 #3 0]{\tikzexternal@filenameprefix tikzoutput/#4-output.pdf}
  \fi
}
\makeatother

\tikzsetexternalprefix{./tikz/}

% -------
% Lengths
% -------

% For N columns, choose \sepwidth and \colwidth such that
% (N+1)*\sepwidth + N*\colwidth = \paperwidth
\newlength{\sepwidth}
\newlength{\colwidth}
\setlength{\sepwidth}{0.0236\paperwidth}
\setlength{\colwidth}{0.4646\paperwidth}

\newcommand{\separatorcolumn}{\begin{column}{\sepwidth}\end{column}}


% ----------------
% Language support
% ----------------
% polyglossia requires fontspec
\usepackage{polyglossia}
\setmainlanguage[variant=british]{english}

% Context-sensitive quotation marks
\usepackage[english=british]{csquotes}


% ------------
% Bibliography
% ------------
\usepackage[%
  sorting=none,%
  style=nature,%
  articletitle=false,%
  autocite=superscript,%
  dateabbrev=false,%
  url=false,%
  isbn=false,%
  backend=biber%
]{biblatex}
\addbibresource{bib/watoc2020poster.bib}


% ---------------
% List of symbols
% ---------------
% Load the glossaries package (after hyperref)
\usepackage[%
  symbols,% create list of symbols
  abbreviations,% create list of abbreviations
  nomain,%
  nonumberlist,%
  nogroupskip,%
  nopostdot%
]{glossaries-extra}

% Create an "ignored" list of symbols that will not be printed out
\newglossary[glignoredl]{ignored}{glignored}{glignoredin}{Ignored Glossary}

% Hyphenated long forms
\glsaddkey
  {hyphenated}        % new key
  {\relax}            % default value if "hyphenated" isn't used in \newglossaryentry
  {\glsentryhyphx}    % analogous to \glsentrytext
  {\Glsentryhyphx}    % analogous to \Glsentrytext
  {\glshyphx}         % analogous to \glstext
  {\Glshyphx}         % analogous to \Glstext
  {\GLShyphx}         % analogous to \GLStext
\newcommand{\GENglspostlinkhook}{%
  \ifglsused{\glslabel}{}{ (\glsentryshort{\glslabel})}\glsunset \glslabel}
\makeatletter
\newcommand\metadef[1]{%
  \expandafter\newcommand\csname gls#1\endcsname{%
    \@ifstar{\csname sgls#1\endcsname}{\csname ngls#1\endcsname}%
  }
  \@namedef{sgls#1}##1{{\let\glspostlinkhook \GENglspostlinkhook\expandafter\csname gls#1x\endcsname*{##1}}}%
  \@namedef{ngls#1}##1{{\let\glspostlinkhook \GENglspostlinkhook\expandafter\csname gls#1x\endcsname{##1}}}%
  \expandafter\newcommand\csname Gls#1\endcsname{%
    \@ifstar{\csname sGls#1\endcsname}{\csname nGls#1\endcsname}%
  }
  \@namedef{sGls#1}##1{{\let\glspostlinkhook \GENglspostlinkhook\expandafter\csname Gls#1x\endcsname*{##1}}}%
  \@namedef{nGls#1}##1{{\let\glspostlinkhook \GENglspostlinkhook\expandafter\csname Gls#1x\endcsname{##1}}}%
  \expandafter\newcommand\csname GLS#1\endcsname{%
    \@ifstar{\csname sGLS#1\endcsname}{\csname nGLS#1\endcsname}%
  }
  \@namedef{sGLS#1}##1{{\let\glspostlinkhook \GENglspostlinkhook\expandafter\csname GLS#1x\endcsname*{##1}}}%
  \@namedef{nGLS#1}##1{{\let\glspostlinkhook \GENglspostlinkhook\expandafter\csname GLS#1x\endcsname{##1}}}%
}
\makeatother

\metadef{hyph}

\makeglossaries
\loadglsentries{symbols/symbols}
\loadglsentries{symbols/acronyms}


% -----
% Boxes
% -----
\usepackage[most]{tcolorbox}
\usepackage{varwidth}
\tcbsetforeverylayer{shield externalize}

\newtcolorbox{violethighlightbox}[2][]{%
  enhanced,
  size=normal,
  height=14.5cm,
  colframe=highlightviolet,
  colback=highlightviolet!10!white,
  colbacktitle=blue!25!yellow!10!white,
  fonttitle=\Raleway\bfseries\large,
  coltitle=highlightviolet!60!black,
  fontupper=\normalsize,
  boxed title size=standard,
  attach boxed title to top center=
  {yshift=-0.25mm-\tcboxedtitleheight/2, yshifttext=-4mm-\tcboxedtitleheight/2},
  boxed title style={%
    boxrule=0.5mm,
    frame code={%
      \path[tcb fill frame] ([xshift=-4mm]frame.west)
      -- (frame.north west) -- (frame.north east) -- ([xshift=4mm]frame.east)
      -- (frame.south east) -- (frame.south west) -- cycle;%
    },
    interior code={%
      \path[tcb fill interior] ([xshift=-2mm]interior.west)
      -- (interior.north west) -- (interior.north east)
      -- ([xshift=2mm]interior.east) -- (interior.south east) -- (interior.south west)
      -- cycle;%
    }%
  },
  title=#2,
  #1
}
\newtcolorbox{orangehighlightbox}[2][]{%
  enhanced,
  size=normal,
  height=14.5cm,
  colframe=highlightorange,
  colback=highlightorange!10!white,
  colbacktitle=blue!25!yellow!10!white,
  fonttitle=\Raleway\bfseries\large,
  coltitle=highlightorange!60!black,
  fontupper=\normalsize,
  boxed title size=standard,
  attach boxed title to top center=
  {yshift=-0.25mm-\tcboxedtitleheight/2, yshifttext=-4mm-\tcboxedtitleheight/2},
  boxed title style={%
    boxrule=0.5mm,
    frame code={%
      \path[tcb fill frame] ([xshift=-4mm]frame.west)
      -- (frame.north west) -- (frame.north east) -- ([xshift=4mm]frame.east)
      -- (frame.south east) -- (frame.south west) -- cycle;%
    },
    interior code={%
      \path[tcb fill interior] ([xshift=-2mm]interior.west)
      -- (interior.north west) -- (interior.north east)
      -- ([xshift=2mm]interior.east) -- (interior.south east) -- (interior.south west)
      -- cycle;%
    }%
  },
  title=#2,
  #1
}
\newtcolorbox{greenhighlightbox}[2][]{%
  enhanced,
  size=normal,
  height=14.5cm,
  colframe=highlightgreen,
  colback=highlightgreen!10!white,
  colbacktitle=blue!25!yellow!10!white,
  fonttitle=\Raleway\bfseries\large,
  coltitle=highlightgreen!60!black,
  fontupper=\normalsize,
  boxed title size=standard,
  attach boxed title to top center=
  {yshift=-0.25mm-\tcboxedtitleheight/2, yshifttext=-4mm-\tcboxedtitleheight/2},
  boxed title style={%
    boxrule=0.5mm,
    frame code={%
      \path[tcb fill frame] ([xshift=-4mm]frame.west)
      -- (frame.north west) -- (frame.north east) -- ([xshift=4mm]frame.east)
      -- (frame.south east) -- (frame.south west) -- cycle;%
    },
    interior code={%
      \path[tcb fill interior] ([xshift=-2mm]interior.west)
      -- (interior.north west) -- (interior.north east)
      -- ([xshift=2mm]interior.east) -- (interior.south east) -- (interior.south west)
      -- cycle;%
    }%
  },
  title=#2,
  #1
}


% ---------------
% Special symbols
% ---------------
% Operators
\DeclareMathOperator{\im}{im}
\DeclareMathOperator{\id}{id}
\DeclareMathOperator{\tr}{tr}
\DeclareMathOperator{\spn}{span}
\DeclareMathOperator{\Ln}{Ln}
\DeclareMathOperator{\diag}{diag}
\DeclareMathOperator{\Sym}{Sym}
\DeclareMathOperator{\Arg}{Arg}